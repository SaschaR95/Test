Gliederung

1.Zielbestimmung
2.Produkteinsatz
3.Produktumgebung
4.Funktionale Anforderung
5.Produktdaten
6.Nichtfunktionale Anforderungen
7.Globale Testfälle
8.Systemmodelle
  -Szenarien
  -Anwendungsfälle
  -Benutzerschnittstelle
  -Dynamische Modelle?
  -Objektmodelle?
9.Glossar


1. Zielbestimmung:
Der Nutzer soll durch das Produkt in der Lage sein, beliebige encodierte Videosequenzen mit ihrem Rohmaterial zu vergleichen.
	
	1.1 Musskriterien
		- gleichzeitiges Anzeigen des Rohmaterials und der encodierten Sequenz
		- Manipulation des Rohmaterials durch einfache Filter
		- Anzeigen der Makroblöcke
		- Auswertung/ Vergleich von Standardwerten wie Bitrate etc.
		- "Frame für Frame" Funktion
		- Grafische Auswertung (Diagramme)		
		
	
	
	1.2 Wunschkriterien
		- gleichzeitiges Anzeigen beliebig vieler Videos, die verschieden manipuliert wurden
		- Hervorheben "kritischer" Stellen
		- Ratingsystem
		- Wiederholtes Encoden
		
		
		
	1.3 Abgrenzungskriterien
		- kein integrierter Encoder (das Rohmaterial wird extern encodiert)


2. Produkteinsatz  -obvious
3. Produktumgebung -obvious

4. Funktionale Anforderungen
	4.1 Manipulation
	-/F10/ S/W Filter ...
	-/F20/ R-Kanal ...
	-/F30/ G-Kanal ...
	-/F40/ B-Kanal
	-/F50/ R-Filter
	-/F60/ G-Filter
	-/F70/ B-Filter
	-/F80/ Weichzeichner (mit Regler?)
	-/F90/ Sepia-Filter
	-/F100/ Horizontalmuster
	-/F110/ Vertikalmuster
	-/F120/ 
	
	 
	
	4.2 Auswertung
	-/F10.2/ PSNR
	-/F20.2/ Bitrate
	-/F30.2/ Kontrast
	-/F40.2/ Auflösung?
	-...
		
		
	
	4.3 Darstellung
	-/F10.3/ Videos abspielen
	-/F20.3/ Diagramme darstellen
	-/...


5.Produktdaten

6.Nichtfunktionale Anforderungen

7.Globale Testfälle

	




8.Systemmodelle
  -Szenarien
  -Anwendungsfälle
  -Benutzerschnittstelle
  -Dynamische Modelle?
  -Objektmodelle?






9.Glossar
	

	
	
